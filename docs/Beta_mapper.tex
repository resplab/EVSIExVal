% Options for packages loaded elsewhere
\PassOptionsToPackage{unicode}{hyperref}
\PassOptionsToPackage{hyphens}{url}
%
\documentclass[
]{article}
\usepackage{amsmath,amssymb}
\usepackage{iftex}
\ifPDFTeX
  \usepackage[T1]{fontenc}
  \usepackage[utf8]{inputenc}
  \usepackage{textcomp} % provide euro and other symbols
\else % if luatex or xetex
  \usepackage{unicode-math} % this also loads fontspec
  \defaultfontfeatures{Scale=MatchLowercase}
  \defaultfontfeatures[\rmfamily]{Ligatures=TeX,Scale=1}
\fi
\usepackage{lmodern}
\ifPDFTeX\else
  % xetex/luatex font selection
\fi
% Use upquote if available, for straight quotes in verbatim environments
\IfFileExists{upquote.sty}{\usepackage{upquote}}{}
\IfFileExists{microtype.sty}{% use microtype if available
  \usepackage[]{microtype}
  \UseMicrotypeSet[protrusion]{basicmath} % disable protrusion for tt fonts
}{}
\makeatletter
\@ifundefined{KOMAClassName}{% if non-KOMA class
  \IfFileExists{parskip.sty}{%
    \usepackage{parskip}
  }{% else
    \setlength{\parindent}{0pt}
    \setlength{\parskip}{6pt plus 2pt minus 1pt}}
}{% if KOMA class
  \KOMAoptions{parskip=half}}
\makeatother
\usepackage{xcolor}
\usepackage[margin=1in]{geometry}
\usepackage{graphicx}
\makeatletter
\def\maxwidth{\ifdim\Gin@nat@width>\linewidth\linewidth\else\Gin@nat@width\fi}
\def\maxheight{\ifdim\Gin@nat@height>\textheight\textheight\else\Gin@nat@height\fi}
\makeatother
% Scale images if necessary, so that they will not overflow the page
% margins by default, and it is still possible to overwrite the defaults
% using explicit options in \includegraphics[width, height, ...]{}
\setkeys{Gin}{width=\maxwidth,height=\maxheight,keepaspectratio}
% Set default figure placement to htbp
\makeatletter
\def\fps@figure{htbp}
\makeatother
\setlength{\emergencystretch}{3em} % prevent overfull lines
\providecommand{\tightlist}{%
  \setlength{\itemsep}{0pt}\setlength{\parskip}{0pt}}
\setcounter{secnumdepth}{-\maxdimen} % remove section numbering
\ifLuaTeX
  \usepackage{selnolig}  % disable illegal ligatures
\fi
\IfFileExists{bookmark.sty}{\usepackage{bookmark}}{\usepackage{hyperref}}
\IfFileExists{xurl.sty}{\usepackage{xurl}}{} % add URL line breaks if available
\urlstyle{same}
\hypersetup{
  pdftitle={beta mapper},
  pdfauthor={Mohsen Sadatsafavi},
  hidelinks,
  pdfcreator={LaTeX via pandoc}}

\title{beta mapper}
\author{Mohsen Sadatsafavi}
\date{2023-12-28}

\begin{document}
\maketitle

\hypertarget{estimating-the-distribution-of-nbs-from-summary-statistics-of-a-model-performance}{%
\subsection{Estimating the distribution of NBs from summary statistics
of a model
performance}\label{estimating-the-distribution-of-nbs-from-summary-statistics-of-a-model-performance}}

Our task is to generate many samples from the joint probability
distribution P(prev, se, sp) from distributions on the performance of
the model in a population.

We assume we have specified probability distribution for the following
parameters:

\begin{itemize}
\item
  Expected prevalence of the outcome in the target population
\item
  c-statistic of the model in the target population
\item
  Calibration slope and intercept for the model in the target population

  \begin{itemize}
  \item
    Calibration slope and intercept refer to A and B in the following
    equation

    \[ logit(P(Y=1 | \pi))=A+Blogit(\pi) \] (the calibration function)
  \end{itemize}
\end{itemize}

\hypertarget{lemma}{%
\subsection{Lemma:}\label{lemma}}

Let \(\pi\) the predicted risk and \(Y\) the corresponding binary
outcome. If \(P(\pi) \sim Beta(a,b)\) then
\(P(\pi | Y=0)\sim Beta(a,b+1)\) and \(P(\pi | Y=1)\sim Beta(a+1,b)\).

Proof: trivial by using Bayes' theorem and the Beta-Bernoulli cojugacy.

We will use the above finding twice: one to simplify the estimation of
c-stat given \(\pi \sim Beta\) to the probability of one Beta RV being
greater than another Beta RV, and again when estimating sensitivity and
specificity of a (potentially miscalibrated) model if the calibrated
risk follows a Beta distribution.

\hypertarget{approach}{%
\subsection{Approach}\label{approach}}

\begin{enumerate}
\def\labelenumi{\arabic{enumi}.}
\item
  We assume the distribution of CALIBRATED RISKS (ie.,
  \(p:= logit(P(Y=1 | \pi))\) follows a Beta distribution in the
  population. This is much more tractable than making assumptions about
  the distribution of \(\pi\). We note that because the logit
  calibration connecting \(p\) and \(\pi\) is monotonic on \(\pi\), our
  knowledge of the c-statistic for predicted risks carries to the same
  distribution for \(p\).
\item
  Because we assume we know the mean and c-statistic, we should be able
  to derive the parameters of this Beta. Given the mean is known, this
  becomes a matter of finding the alpha parameters of the Beta
  distribution that gives rise to a given c-statistics.
\end{enumerate}

Given the above lemma, we will be dealing with Beta distribution all
along. TODO: show that a=f(c-stat) is monotonical for a given mu?

\begin{enumerate}
\def\labelenumi{\arabic{enumi}.}
\setcounter{enumi}{2}
\tightlist
\item
  We estimate the sensitivity and specificity of the model at a given
  threshold given the above-mentioned Beta, as well as calibration
  intercept and slope
\end{enumerate}

\[ se=P(\pi \ge z | Y=1)  \], \[ sp=P(\pi \lt z | Y=0)  \].

Again, given the lemma, these probabilities follow a Beta distribution,
so this is easy to calculate. For example,
\(se=P(\pi \ge z | Y=1)=P(logit(\pi)>logit(z) | Y=1)=P(A+Blogit(\pi)>A+Blogit(z) | Y=1)=P(logit(p)>A+Blogit(z) | Y=1) = P(p>expit(A+Blogit(z)) | Y=1)\)
(assuming B\textgreater0 which is safe).

And we do know \(P(p | Y=1)\) given the Lemma.

END OF DOCUMENT

\end{document}
